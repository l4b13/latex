\documentclass[a4paper,12pt]{article}

%%% Работа с русским языком
\usepackage{cmap}                   % поиск в PDF
\usepackage{mathtext}               % русские буквы в формулах
\usepackage[T2A]{fontenc}           % кодировка
\usepackage[utf8]{inputenc}         % кодировка исходного текста
\usepackage[english,russian]{babel} % локализация и переносы

%%% Страница
\usepackage{extsizes} % Возможность сделать 14-й шрифт
\usepackage{geometry} % Простой способ задавать поля
    \geometry{top=20mm}
    \geometry{bottom=20mm}
    \geometry{left=20mm}
    \geometry{right=15mm}
\usepackage{fancyhdr} % Колонтитулы

\usepackage{setspace} % Интерлиньяж
% \onehalfspacing % Интерлиньяж 1.5

\pagestyle{empty}

\usepackage{hyperref}
\usepackage[usenames,dvipsnames,svgnames,table,rgb]{xcolor}
\hypersetup{                % Гиперссылки
    unicode=true,           % русские буквы в раздела PDF
    pdftitle={Заголовок},   % Заголовок
    pdfauthor={Автор},      % Автор
    pdfsubject={Тема},      % Тема
    pdfcreator={Создатель}, % Создатель
    pdfproducer={Производитель}, % Производитель
    pdfkeywords={keyword1} {key2} {key3}, % Ключевые слова
    colorlinks=true,        % false: ссылки в рамках; true: цветные ссылки
    linkcolor=black,        % внутренние ссылки
    citecolor=black,        % на библиографию
    filecolor=blue,         % на файлы
    urlcolor=blue           % на URL
}

\usepackage{indentfirst} % отступ после заголовка

\usepackage{multicol} % Несколько колонок

\usepackage{enumitem}

\usepackage{csquotes} % Инструменты для ссылок

%%% Работа с картинками
\usepackage{graphicx}  % Для вставки рисунков
\graphicspath{{images/}}  % папки с картинками
\setlength\fboxsep{3pt} % Отступ рамки \fbox{} от рисунка
\setlength\fboxrule{1pt} % Толщина линий рамки \fbox{}
\usepackage{wrapfig} % Обтекание рисунков и таблиц текстом

\title{resume}
%\author{Vladislav Obudov}
%\date{May 2022}


\begin{document}

\begin{wrapfigure}{r}{0pt}
    \includegraphics[scale=0.7]{004.jpg}
\end{wrapfigure}

\begin{center}
    {\largeРезюме}
\end{center}



ФИО: Обудов Владислав Антонович

Дата рождения: 14.09.1999

Место рождения: город Якутск

Семейное положение: холост

Гражданство: РФ

Телефон: +7 (911) 796-73-55

E-mail: vlad\_dkfl@bk.ru

Цель трудоустройства: прохождение практики\\

Образование:

2006 г. по 2010 г. МОБУ «СОШ №30 им. В. И. Кузьмина»

2010 г. по 2017 г. МОБУ «Физико-технический лицей им. В. П. Ларионова»

2020 г. по наст. время ФГАОУ ВПО «СВФУ им. М. К. Аммосова», направление Фундаментальная информатика и информационные технологии, дата окончания 2024 г.\\

Опыт работы:

2022 г. по наст. время --- ГАУ ДО РС(Я) «МАН РС(Я)» --- диспетчер образовательного учреждения

Обязанности:
\begin{itemize}[noitemsep,topsep=0pt,parsep=0pt,partopsep=0pt]
    \item работа с электронными документами
    \item помощь в организации курсов и олимпиад по информатике\\
\end{itemize}

Дополнительная информация:

Владение языками:

\begin{itemize}[noitemsep,topsep=0pt,parsep=0pt,partopsep=0pt]
    \item якутский - родной
    \item русский - в совершенстве
    \item английский - выше среднего
\end{itemize}

Водительское удостоверение: категория B, стаж 1 год

Возможные командировки: нет

Владение ПК: уверенный пользователь\\

О себе:

Без вредных привычек

Оканчиваю 2 курс направления ФИИТ в СВФУ предварительно на отлично. Изучал: Java, C, C++, C\#, Pascal. Помимо этого ознакомлен с СУБД PostgreSQL, немного с MongoDB, архитектурой MIPS. Некоторые лабораторные работы: \href{https://github.com/l4b13/}{github.com/l4b13}

Хобби: видеоигры, прогулки на велосипеде, лыжах, спортивное программирование

Достижения в олимпиадах:

2017 г. Республиканская командная олимпиада школьников по программированию --- диплом 1 степени, 2 место

2020 г. Открытая олимпиада ИМИ по программированию, категория C (студенты 1 курсов ИМИ) --- диплом 1 степени, 1 место

2021 г. Открытая Республиканская командная олимпиада по программированию ИМИ, категория B (студенты 1-2 курсов ИМИ) --- диплом 1 степени, 1 место

2022 г. Открытая Республиканская командная олимпиада по программированию ИМИ, категория B (студенты 1-2 курсов ИМИ) --- диплом 1 степени, 1 место

От практики ожидаю набраться опыта в решении реальных задач/проблем и познакомиться с талантливыми людьми

\end{document}